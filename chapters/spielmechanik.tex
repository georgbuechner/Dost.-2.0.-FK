\section{Blackouts}\label{sec:blackouts}

\emph{Fabel: Da die allermeisten Menschen heutzutage \aclp{lsc}\acused{lsc} (\ac{lsc}s)
eingebaut haben, stirbt \ac{m} nicht, sondern bekommt einen >>Blackout<<.
Dieser kann bis zu einem Tag andauern, danach weiß M nichts mehr, das
Gedächtnis ist vollständig ausgelöscht, nur motorische und soziale Fähigkeiten
sind noch vollständig intakt (alles was M als >>zweite Natur<< bezeichnen
würde).}\\\\
%
Wenn die Spieler*in nach einem Blackout wieder >>erwacht<<, findet sie sich in einem 
>>neuen Menschen<<. 
In der App, mit der die \ac{lsc}s verbunden sind und in der diese (wie auch die 
Spieler*innen und Menschen) verwaltet werden, findet die Spieler*in sich in
einem anderen Account eingelogt, sieht ihren neuen Namen und erhält durch die
geleakte \nameref{sec:geheimdienstakte} ihre neue Identität (so wie bereits zu Beginn des
Spieles, das Aufwachen nach dem Tod ist identisch mit dem ersten Erhalt des
Charakters vor dem Spiel). 
Außerdem finden sich in den Tagebucheinträgen aktuelle Informationen, über die
jüngsten Erlebnisse, Ambitionen und Gedanken. 
Selbstverständlich sieht die neue Spieler*innen auch alle Nachrichten Posts
etc., die die Spieler*in zuvor als ihr Charakter geschrieben hat.\\\\
%
Durch diese Blackout-Dynamik sind die Spieler*innen tendenziell gezwungen
fleißig Tagebucheinträge zu schreiben, damit sie (bzw.~die nächste Spieler*in) sich 
später daran erinnern können. 
Zur Frage, wie dieser Zwang Beginn des Spiels klargemacht werden könnte:
IT-Motivation.~Durch die Fabel wird den Spieler*innen klar, dass
Tagebuch-Schreiben für jeden Menschen dieser Welt essenziell ist: Einen Blackout
zu bekommen, ohne zuvor intensiv Tagebuch geführt zu haben, bedeutet Teile
seines Lebens und Erfahrungen unwiderruflich zu verlieren.
OT-Motivation.~Zusätzlich könnte es Belohnungssysteme geben: Wer Tagebuch
schreibt, erhält XX Geld.\footnote{
  Dies könnte das Konzept einer Art Versicherungsgesellschaft sein:
  \qq{Versicherte} zahlen einen monatlichen Beitrag an die
  Versicherungsgesellschaft, mit jedem Tagebucheintrag steigt der Zinssatz.
  Motivierte Schreiber*innen holen dadurch mehr Geld raus und pushen sich, auf
  die Tagebucheinträge zu achten. 
  Das Geschäftsmodell beruht auf der Annahme, dass viele Menschen zu
  faul sind und nicht genug schreiben, um finanziell mehr herauszuholen, als sie
  reingesteckt haben, das Angebot aber dennoch nutzen, um die vage Hoffnung
  darauf aber nicht aufzugeben (so wie manche auch heute ihr Fitnessstudio oder
  Museums Abonnement weiterführen, mit dem Gedanken: >>So bin ich wenigstens
  motiviert, bestimmt nächste Woche zu gehen.<< Oder: >>Na, wenn ich mein Abo
  kündige, gehe ich ja sicher nicht mehr hin. Ich bin doch nicht bescheuert!<<)
}
Allerdings ist die Frage, wie diese Zwang bereits für den ersten Charakter den M
erhält klar ist.\\\\
%
\emph{(Idee: Während dem Blackout ist die Spieler*in nicht in einen Charakter
eingeloggt, bzw.~mit dem Tod wird sie sofort ausgeloggt. Damit verändert sich
das Layout der App (z.B.~Hintergrundfarbe, Schriftart etc.) und sie sieht 
die Option, auch hier Nachrichten schreiben und Kontakte hinzufügen zu können. 
Dadurch sollen sie erahnen können, dass es auch ein Spiel \qq{hinter} dem Spiel
geben könnte. Nur sehr Spieler*innen haben bereits beim ersten Blackout Kontakte
und sogar Nachrichten. Solche Spieler*innen sind diejenigen tendenziell die
Storyline der Surrealistinnen entdecken.)}

\section{App} 
Die App hat einige zentrale Funktionen für das Spiel:
\begin{itemize}
  \item Charakterverteilung und Einsicht in die eigene Rolle (Geheimdienstakte
    und Tagebuch)
  \item in-game Kommunikation über Posts (Twitter/ Reddit ähnlich) und Direkt
    Messages. 
  \item Hacken anderer Spieler*innen (Nachrichten abfangen, lesen, aber durch
    die Verbindung der App mit den \ac{lsc}s auch \nameref{sec:blackouts}
    herbeiführen).
  \item Verwaltung von Finanzen (E-Pay System)
  \item Darstellung (und Steuerung) des Krieges
\end{itemize}

\section{Krieg}
Der Krieg wird auf der App dargestellt. 
Langsam verwandelt sich dort auto- bzw.~Zufalls-generiert die Bewegung der
Truppen, Siege einzelner Gefechte, Eroberungen neuen Gebietes. 
Die Darstellung ist abstrakt (blaue, roter, größere oder kleinere Punkte auf
einer Karte) und einige Gebiete oder Daten des Kriegsgeschehens bleiben
verborgen, bzw.~sind nicht allen Menschen zugänglich. 
So sehen die Mitglieder der \nameref{ssec:lilie} potenziell das ganze
Kriegsgeschehen, dem roten Block bleiben Ausschnitte des Gebiets des blauen, dem
blauen Block hingegen Ausschnitte des Gebiets des roten Blocks verborgen. Auch
innerhalb der einzelnen Fraktionen gibt es aufgrund interner Hierarchien
unterschiedlichen Zugriff.\\\\
%
Interaktionen auf der App können das Kriegsgeschehen beeinflussen. 
Jeder Like auf einen eigenen Post, jeder neue Follower, jeder neue
Tagebucheintrag zieht den Zufallsgenerator etwas mehr auf die Seite der eigenen
Fraktion (die \nameref{ssec:lilie} verlangsamt die Zufallsgeneration und führt so zu
weniger Toten).\footnote{
  Mit der Gewichtung muss experimentiert werden. Ein Follower zählt vermutlich
  mehr, als ein Like.
}
\emph{Es soll den Spieler*innen potenziell möglich sein, zu erkennen, dass die
eigenen actions in der App Einfluss auf den Krieg haben.}
Außerdem kann das Kriegsgeschehen durch die Verhandlungen verändert werden,
es z.B.~entschleunigen, oder Vorteile für den eigenen Block erreichen.\\\\
% 
So weiter das Kriegsgeschehen \emph{voranschreitet}, desto weiter steigt die
Atombombendrohung, bis beim Ausbruch des Atomkrieges das Spiel über alle
Spieler*innen gesiegt hat: \emph{Sieg des Spiels über die Spielerinnen.
Auslöschung von Geist und Natur.}

\section{Geheimdienstakte}\label{sec:geheimdienstakte}

\emph{Fabel: Vor eins-komma-zwei Jahren im Jahre XXII NnZ wurde (vermutlich von
einer verdeckt operierenden Schweizer Agent*innen-Assoziation) eine
Geheimdienstakte geleakte, genauer ein zusammengestelltes Dokument verschiedener
Geheimdienstakte beider Blöcke geleakt. Es wurde viel über den Grund dieses
Leaks, bzw.~die Ambitionen der Assoziation diskutiert. Möglich ist, dass sie
damit beide Blöcke als verletzlich zeigen wollten, um Friedensverhandlungen zu
beschleunigen. Dass \ac{dost2} überhaupt stattfinden sollte, führen einige auf
dieses Leak zurück.}
