Die \ac{dost2} ist eine Friedenskonferenz, auf der Vertreterinnen der beiden
kriegführenden Blöcke (unter Leitung der USA -- \nameref{ssec:blauer_block} --
einerseits und der Chinas -- \nameref{ssec:roter_block} -- andererseits) auf
einem Schloss in der neutralen Schweiz (OT: in Brandenburg, Schloss Poggelow)
zusammenkommen, um über den Frieden, bzw.~den weiteren Verlauf des Krieges zu
verhandeln.
Allerdings ist -- bis auf einige Liberale -- niemand wirklich an \emph{Frieden}
interessiert, stattdessen dominieren vor allem persönliche oder nationale
(bzw.~nationalistische) Motive. Daher gleicht das Ambiente (ein Schloss in den
malerischen Bergen der Schweiz) mehr einer hochkarätigen Feier als einer
ernstzunehmenden Konferenz: anregende Musik, ausgezeichneter Alkohol, reinste
Drogen, genüssliche Kunst und Lust-versprechendes Theater bestimmen das
Programm. Gewiss es muss auch verhandelt werden, aber die allermeisten Verträge
werden ohnehin im Geheimen, abseits der Versammlungssäle im Pool oder bei einem
Glas Whiskey im Rauchersalon getroffen. Dennoch: Das erwartete Vergnügen der
geladenen Gäste wird sich in einen schaurigen Schrecken verwandeln, denn es
lauern gleich zwei Gefahren: die revolutionäre und die surrealistische.\\\\
%
Im Schatten der Neutralität wurde der Schweizer Geheimdienst, der \ac{ndb}, von
einer kleinen aber sorgfältig agierenden revolutionären Assoziation
unterwandert und es sollte gerade der \ac{ndb} sein, der \ac{dost2} ausrichtet.
Die Friedenskonferenz findet also >>in der Höhle des Löwen<< statt.\\\\
%
Die zweite Gefahr ist subtiler, noch weniger zu fassen und erst recht zu
erahnen. Durch die technische Entwicklung wurde es möglich, den Tod durch einen
vorübergehenden Blackout (siehe: \nameref{sec:blackouts}) zu umgehen.
Dieser ungeheure Sieg des Geistes über die Natur bringt aber schleichend das von
ihm Unterdrückte wieder hervor: 
In der Zeit, die der Blackout andauert, formt sich langsam ein eigenständiges
Bewusstsein, das bald selbstständige Ziele und Pläne schmiedet, die Herrschaft
des Geistes über die Natur rückgängig zu machen.\\\\
%
Aus dieser Konstellation ergeben sich potenziell fünf verschiedene
>>Siegeskonstellationen<<: 
\begin{itemize} 
  \item[] Sieg einer der Blöcke, \emph{Herrschaft des Geistes}.
  \item[] >>Kalter<< Sieg der Revolutionäre, \emph{Herrschaft des Geistes}.
  \item[] >>Absoluter<< Sieg des Surrealismus, \emph{Herrschaft der Natur}.
  \item[] Gemeinsamer Sieg von Revolution und Surrealismus, \emph{Symbiose von
    Geist und Natur}.
  \item[] Sieg des Spiels über die Spielerinnen: Ausbruch des Atomkriegs,
    \emph{Auslöschung von Geist und Natur}.
\end{itemize}
Das Spiel wird dabei begleitet von einer App, auf der das aktuelle
(randomisiert sich verändernde) Kriegsgeschehen visualisiert ist, die Menschen
kommunizieren, sich aber auch gegenseitig \qq{hacken} und dadurch Blackouts
auslösen können. 
Außerdem laufen die Finanzgeschäfte (die eng mit dem Kriegsgeschehen verbunden
sind) über diese App. Zugleich aber: Jede Interaktion auf der App verändert das
Kriegsgeschehen.
Welche Interaktionen in der App welche Änderungen im Kriegsgeschehen auslösen,
kann von den Spieler*innen potenziell herausgefunden und zu ihren Gunsten
genutzt werden.
