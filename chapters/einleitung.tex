\ac{dost2} ist eine Friedenskonferenz, auf der Vertreterinnen der beiden
kriegführenden Blöcke (unter Leitung der USA einerseits und der Chinas
andererseits) auf einem Schloss in der neutralen Schweiz zusammenkommen, 
um über den Frieden, bzw.~den weiteren Verlauf des Krieges zu verhandeln.
Allerdings ist -- bis auf einige Liberale -- niemand wirklich an \emph{Frieden}
interessiert, stattdessen dominieren vor allem persönliche oder nationale
(bzw.~nationalistische) Motive. Daher gleicht das Ambiente (ein Schloss in den
malerischen Bergen der Schweiz) mehr einer hochkarätigen Feier als einer
ernstzunehmenden Konferenz: Musik, Alkohol, Drogen, Kunst, Theater bestimmen das
Programm. Gewiss es muss auch verhandelt werden, aber die allermeisten Verträge
werden ohnehin im Geheimen, abseits der Versammlungssäle im Pool oder bei einem
Glas Whiskey im Rauchersalon getroffen. Dennoch: Das erwartete Vergnügen der
geladenen Gäste wird sich in einen schaurigen Schrecken verwandeln, denn es
lauern gleich zwei Gefahren: die revolutionäre und die surrealistische.\\\\
%
Im Schatten der Neutralität wurde der Schweizer Geheimdienst, der \ac{ndb}, von
einer kleinen aber sorgfältig agierenden revolutionären Assoziation
unterwandert und es sollte gerade den \ac{ndb} sein, der die \ac{dost2}
ausrichtet. Die Friedenskonferenz findet also >>in der Höhle des Löwen<<
statt.\\\\
%
Die zweite Gefahr ist subtiler, noch weniger zu fassen und erst recht zu
erahnen. Durch die technische Entwicklung, wurde es möglich, den Tod zu umgehen
und ihn durch einen vorübergehenden Blackout (siehe: \nameref{sec:blackouts})
\qq{abzumildern}. 
Dieser ungeheure Sieg des Geistes über die Natur bringt aber schleichend sein
Gegenteil hervor: In der Zeit, die der Blackout andauert, formt sich langsam ein
eigenständiges Bewusstsein, das bald selbstständige Ziele und Pläne schmiedet,
die Herrschaft des Geistes über die Natur rückgängig zu machen.\\\\
%
Aus dieser Konstellation ergeben sich potenziell fünf verschiedene
>>Siegeskonstellationen<<: 
\begin{itemize} 
  \item[] Gewinnt einer der Blöcke, kann dieser mit dem dadurch errungenen Boom,
    die Herrschaft des Geistes über die Natur für alle Zeiten verabsolutieren.
  \item[] >>kalter<< Sieg der Revolutionäre, Herrschaft des Geistes
  \item[] >>absoluter<< Sieg der Natur, Herrschaft der Natur
  \item[] Gemeinsamer Sieg von Revolution und Natur. 
  \item[] Sieg des Spiels über die Spielerinnen: Ausbruch des Atomkriegs
\end{itemize}
Das Spiel wird dabei begleitet von einer App, auf der das aktuell
(randomisiert verändernde) Kriegsgeschehen visualisiert ist, die Menschen
kommunizieren, sich aber auch gegenseitig Haken können. Außerdem laufen die
Finanzgeschäfte (die eng mit dem Kriegsgeschehen verbunden sind) über diese App.
Zugleich aber: Jede Interaktion auf der App verändert das Kriegsgeschehen.
