Ein zentraler Aspekt von \ac{dost2} ist die kollektive Entwicklung, die dabei
selbst die tief liegende Konflikt-Linie des \aclp{fdl} (\acs{fdl}) \acused{fdl}
-- Geist gegen Natur -- thematisiert: Einem Kunstwerk ähnlich, in welchem Form
(Geist) und Material (Natur) ineinandergreifen, werden die einzelnen Elemente
des \ac{fdl}s zwar kollektiv entwickelt aber von \creators{} in eine Form
gebracht. Ob sich im Spiel also Geist oder Natur durchsetzen, die Surrealisten
sich mit den Revolutionären verbinden oder beide von den Blöcken vernichtet
werden, ist auch die Frage, ob sich das Kollektiv gegen \emph{Alex \& fux}
durchsetzt, oder diese sich gegen jenes, schließlich, ob beide eine Symbiose
eingehen. 

Es sollen möglichst wenig Menschen mehrfach eingebunden werden, um ein hohes
secrecy-level zu ermöglichen.

\section{Schreiben der Charaktere}
Insgesamt sind alle Charaktere (bis auf die Surrealisten) einer der drei
Gruppen: Amerika, China oder den Revolutionären zugeteilt. Diese sollen vom
Kollektiv geschrieben werden: 
\begin{itemize} 
  \item Amerika: Dennis und Jojo 
  \item China: Anna und ???
  \item Revolutionäre: Martin und Marie 
\end{itemize}
Die Surrealisten werden von \creators{} entwickelt.

\section{Theater // Audiowalk}
Der Audiowalk soll ein erotisches Lustspiel darstellen und von \emph{fux} und
Mo (Bremen) entwickelt werden. 

\section{Musik}
Insgesamt sollen drei DJs anwesend sein, die Musik auflegen. Die DJs gehören
jeweils einen der beiden Blöcken an und sind geladene Gäste (sie sitzen nicht am
Verhandlungstisch).
\begin{itemize} 
  \item Leon 
  \item Malou 
  \item Ella
\end{itemize}

\section{Gemälde} 
Auf der Friedenskonferenz soll in der Galerie Kunst ausgestellt werden.
\qq{Werke} könnten potenziell von 
\begin{itemize} 
  \item Flo, Ali (Collagen) 
  \item Anna, Michelle (Öl)
  \item Svenja (Aquarell) 
  \item Diva (Fotographie)
\end{itemize}

\section{App} 
Die App soll von Simon und Sophia unter Mitwirkung von \creators{} entwickelt
werden.

\section{Awareness}
Awareness gehört auch mit zum Spiel und sollte von mindestens zwei Menschen
gemacht werden. Es gilt auch einige spieltechnische Entscheidungen in enger
Rücksprache mit dem Awareness-Team umzusetzen (z.B.~Thema Drogen, Alkohol,
Darstellung von Sexualität).
