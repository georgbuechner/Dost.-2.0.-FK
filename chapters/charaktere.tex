Die Charakterbeschreibungen setzten sich aus zwei Teilen zusammen: A) einem
Eintrage in einer geleakten \nameref{sec:geheimdienstakte}; B) den letzten
Tagebucheinträgen. 
Während die Geheimdienstakte die allgemeinsten Daten enthält: Name, Alter, 
Geburtsort, Partei-(/Fraktions-)Zugehörigkeit, pottenziel andere, für die
Öffentlichkeit interessante Aktivitäten oder Ziele (z.B.~eine Affäre,
wirtschaftliche/ politische Ambitionen, bedeutende Freund- oder Feindschaften
\dots)\footnote{
  Selbstverständlich kann es sein, dass einige Daten in der Akte fehlen, weil
  sie unbekannt sind, oder geschwärzt wurden, weil sie einer höheren
  Sicherheitsstufe zugeordnet wurden.
}, bestehen die Tagebucheinträge aus intimeren und aktuelleren
Informationen, welche die Geheimdienstakte nicht abbilden kann.\\\\
%
\emph{Hinweis: Durch die \nameref{sec:blackouts} muss es mehr Charaktere, als
Spieler*innen geben. Vllt.~ca.~20\% mehr.}

\section{Fraktionen}\label{sec:fraktionen}
Es gibt insgesamt vier Fraktionen: die zwei Blöcke (\nameref{ssec:blauer_block} unter
Führung der USA, \nameref{ssec:roter_block} unter Führung Chinas), die
\nameref{ssec:lilie} und die \nameref{ssec:surrealisten}.
Die zwei Blöcke umfassen große Teile der internationalen Nationen und befinden
sich seit sieben-komma-drei Jahren in einem verheerenden dritten Weltkrieg.
Die Lilie ist eine emanzipatorisch-revolutionäre Assoziation. Obwohl sie
hauptsächlich in der Schweiz aktiv ist, kommen ihre Mitglieder aus allen Teilen
der Welt selbst aus der vierten Welt.
Die Surrealistinnen hingegen sind das was zwischen Mensch und Natur schwebt, was
am Geist mehr ist als bloß Geist, also auch Natur und was an Natur mehr ist als
bloß Natur, als auch Geist. Das, was nach einem Blackout im alten Körper bleibt,
was die App nicht völlig beherrschen kann, was uns transzendiert. 
Nur wenige Spieler*innen werden die surrealistische Realität hinter der Welt
erkennen\dots\\\\
%
Alle Spieler*innen gehören zu Beginn einer der ersten drei Fraktionen an. Ob
eine Person zur Surrealistin wird, ergibt sich erst während des Spiels, kann
aber in einigen Charakteren als mehr, in anderen als weniger wahrscheinlich
angelegt sein.

\subsection{Blauer Block}\label{ssec:blauer_block}

\subsection{Roter Block}\label{ssec:roter_block}

\subsection{Lilie}\label{ssec:lilie}
Die Lilie ist eine emanzipatorisch-revolutionäre Assoziation, die in einer
langwierigen Mission über mehrere Jahre und höchster Vorsicht Teile des \ac{ndb}
unterwandert haben. Präziser: diejenigen Teile, die mit der Durchführung
und Sicherung von \ac{dost2} beauftragt waren.

\subsection{Surrealistinnen}\label{ssec:surrealisten}

